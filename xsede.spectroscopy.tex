\documentclass[11pt,preprint]{aastex}

%compact captions %%%%%%%%%%%%%%%%%%%%%%%%%%%%%%%%%%%%%%%%%%%%%%%%%%%
%\usepackage[font=small,labelfont=bf, textfont=it, justification=justified]{caption}
%\DeclareCaptionFormat{ruleformat}{\baselineskip0.2cm\hrulefill\\\noindent#1#2#3{\hrulefill}}
%\captionsetup[figure]{format=ruleformat}
\setlength{\textfloatsep}{0pt}
%\setlength{\abovecaptionskip}{-10pt}
\setlength{\floatsep}{2pt}

\usepackage{verbatim}
\usepackage{rotating}
\usepackage{lscape}
\usepackage{subfigure}
\usepackage{natbib}
\usepackage{arydshln}

\usepackage[colorlinks=true, linkcolor=blue, urlcolor=blue, citecolor=blue, pdftex, bookmarks=true, linktocpage=true, hyperindex=true]{hyperref}
\usepackage{breakurl}

\providecommand{\xxxx}{{\color{red} ~XXXX~}}
\providecommand{\xsede}{XSEDE~}
\providecommand{\eg}{e.g.,~}
\providecommand{\ie}{\textit{i.e.},~}

\begin{document}

\section*{\Large Toward Measuring the Stellar Initial Mass Function in the Distant Universe}


\begin{center}
Principle Investigator: Dr. Daniel Weisz\\
University of Washington
\end{center}

{\small
{\bf abstract:}
Four years ago, we initiated the single most ambitious mapping project
ever taken with the Hubble Space Telescope: to produce a map of every
luminous star in the nearby Andromeda Galaxy (M31) in ultraviolet,
optical, and near-infrared wavelengths. These $>$117,000,000 stars
encode the history of the galaxy and the rich, detailed physics that
drives stellar and galactic evolution. Using our first successful XSEDE
program, our collaboration pioneered spatially mapping the recent
history of star formation throughout M31, making a ``movie" of the past
star formation of the galaxy during the last half a Gigayear.  We now
propose to extend this movie back to the oldest ages by using a
forward modeling approach that reproduces the observed colors and
brightnesses of stars in small cells across the galaxy. We have
developed both a parameterized model that responds to the added
complexity of extending our age coverage, and an optimized search
strategy that allows us to efficiently scan parameter space.  We have
matched this project to specific XSEDE resources, have meticulously
optimized our strategy, and have benchmarked our specific analysis
pipeline.  Combining the computational power of XSEDE with the
unprecedented number of stars cataloged by our large area survey will
allow us to derive the most precise spatially-resolved star formation
history of M31 throughout cosmic time, probing the earliest epochs of
galaxy formation }


\section{Background \& Motivation}
\label{sec:overview}

A primary goal of astrophysics is to decipher the history of the universe.  Among the most promising approaches has been through the study of galaxies across cosmic time. A galaxy's brightness and color encodes information about its age, rate of star formation, and metal content.  By surveying large collections of galaxies at different cosmic look back times, we can measure how the universe has changed as a function of time.  

The use of galaxies as cosmological tools has proliferated over the past two decades.  Large, systematic surveys such as Sloan Digital Sky Survey (SDSS) and the Hubble Ultra-Deep field have provided precise measurements for millions of galaxies back to the genesis of galaxy formation.  The success of these programs has motivated the next-generation of astronomical facilities such as the Large Synoptic Survey Telescope, which will survey more than 1 billion galaxies, and the James Webb Space Telescope, which promises to directly detect the first galaxies.

Yet, despite the wealth of galaxy observations, how we translate their observed light into physical quantities (e.g., mass, metallicity) remains rudimentary.  Typical observations of a distant, unresolved galaxy consist a handful of fluxes measured at different wavelengths that sample the galaxy's spectral energy distribution (SED).  Physically motivated models of galaxies contain many more parameters than typical data points, resulting in degeneracies between the parameters. To mitigate these degeneracies, key assumptions about a the form of a galaxy's star formation history (SFH), dust content, and stellar initial mass function (IMF) are necessary.  

The validity of these assumptions can usually be tested using detailed observations of resolved galaxies.  For example, the dust distribution of hundreds of local galaxies have been used to explore multiple forms of dust laws, which appear to provide good descriptions of more distant systems.

However, similar tests have not been possible with the stellar IMF.  The stellar IMF describes the relative distribution of stellar masses for a group of stars born at the same time.  In principle, measuring the IMF is a straight-forward exercise in counting the relative number of individual stars at different masses in a star cluster, i.e., a co-eval stellar population.  However, there are two important factors that complicate this measurement in practice.  The first is the need for resolved stars.  Even the high-angular resolution of the Hubble Space Telescope can only resolve individual stars for IMF measurements in the nearest few galaxies.  Second, measuring the IMF in more distant galaxies where individual stars cannot be resolved is challenging due to the strong degeneracy between SFR and the IMF.  As a result, state-of-the-art interpretation of distant galaxies still necessarily relies on the IMF measured in the very different, very local universe.

The consequences of relying on a local IMF to understand interpret the light of distant galaxies is unclear.  On one hand, many theories of star formation predict that the the IMF should be influenced by local star formation conditions (e.g., gas density, star formation intensity).  Thus, the vigorous star formation in galaxies of the early universe may have different IMFs that those found locally.  On the other hand, direct star counts in the very local universe do not show evidence of an IMF that depends systematically on environment. This could either be because the IMF is ``universal'' or because the local universe has a limited dynamic range of star-forming environments.  Extensions of IMF studies to the slightly less local universe has providing tantalizing hints that the IMF does vary with environments; the integrated properties of nearby, low-mass galaxies are consistent with an environmentally dependent IMF.  However, this interpretation makes strong assumptions about the galaxy's SFH, which may also be capable of producing this effect.  In short, the IMF is challenging to measure and remains one of the largest sources of uncertainty in use of galaxies as cosmological tools.


\section{Measuring the High-Mass IMF in the Distant Universe}

Here, we propose a qualitatively new method to measure the IMF in unresolved, distant galaxies.  Our focus is on the IMF for stars more massive that 1 \msun\ (the ``high-mass IMF''), which are entirely responsible for the SEDs of all star-forming galaxies.

Our approach to measuring the IMF in distant galaxies is to use the integrated spectra of young star clusters. The full optical spectrum of a young ($<$ 50 Myr) star cluster contains significant information about the slope of the high mass IMF.  The shape of the blue continuum and size of spectral features (e.g., nebular emission, the Balmer jump) are sensitive to the slope of the high mass IMF, whereas conventional analysis of normalized spectrophotometric indices (e.g., Lick indices) are known to be insensitive to IMF variations \citep[e.g.,][]{kol08}.  At the same time, by definition, star clusters are single age populations, which mitigates the SFH-IMF degeneracies that plague traditional integrated light studies of galaxies.  We provide a more detailed description of this technique in \S \ref{XXX}.  

\subsection{Validating Spectroscopic IMF constraints in M31}

Our program is being conducted in two phases.  The first phase is development and validation of the methodology.  The technical details of the code development are described in \S \ref{XXX}.  

Following successful trials of the code on simulated data, we are now validating our spectroscopic IMF constraints against `gold standard' IMF measurements from resolved star counts.  As part of the Panchromatic Hubble Andromeda Treasury, a 4 year Hubble Space Telescope campaign to resolve stars in our closest neighbor, the Andromeda galaxy, PI Weisz has measured the IMF in $>$500 resolved star clusters.  At the same time, we have also acquired high fidelity spectra of $\sim$ 50 of these same star clusters using the 6.5m Multi-Mirror Telescope in Arizona and the 10m Keck Telescope in Hawaii.  

Using our spectroscopic analysis tools, we are measuring the physical properties (e.g., age, mass, IMF) of each cluster from its integrated spectrum and comparing it to the ground-truth values derived from resolved star counts.  Consistency between these two sets of parameters will provide key validation of the spectroscopic technique. 





\subsection{The First High-Mass IMF Measurement Outside the Local Group}

NGC~628 is an ideal target for a pilot study.  It is a nearby (D $\sim$ 9 Mpc), face-on spiral galaxy that hosts a rich young cluster population (Larsen et al. 1999). It has a recent star formation rate (SFR) of $\sim$ 2- 3 \msun /yr (Sanchez et al.\ 2011) making it an excellent Milky Way analog.  Further, NGC~628 hosts the highest frequency of recent supernovae (SNe; 3 in the last decade;  2002ap, 2003gd, 2013ej) of any galaxy within 10 Mpc.  This high recent SNe rate may indicate that NGC 628 has a higher than normal expected massive star population, which could be due to a recent burst or the possibility of a top-heavy IMF that could also produce a larger number of massive SNe progenitors relative to a standard IMF.

Studying the IMF in NGC~628 provides a nice comparison for our ongoing IMF work in M31.  Compared to M31, NGC~628 is more actively forming stars 3 vs. 0.5 \msun\ / yr), is located in the blue cloud instead of the green valley, and has clear undisturbed spiral morphology, which is more typical than the star-forming `rings' of M31.  NGC~628 also provides a better observational target for this study. M31 hosts only a handful of massive young clusters ($\sim$ 10$^4$ \msun).  Lower mass clusters are subject to stochastic sampling of the IMF effects that are challenging to efficiently include spectrophotometric fitting.  Measuring the IMF in higher mass clusters mitigates stochastic IMF sampling, providing more secure IMF determinations.

This program is the first step in a larger planned LRIS survey to systematic search for environmentally sensitivity of the upper IMF.  NGC~628 is an excellent template in which we can learn about the IMF in a Milky Way analog, demonstrate the capabilities of this approach to the broader community, and efficiently enable us to refine our reduction and analysis pipelines (e.g., LRIS exposure times, HST photometric reductions, etc) before pursuing a larger systematic LRIS study of the IMF.



\section{Potential Science Impact}



\section{Computational Procedure}


\section{Code Description}

Information about the properties of a stellar cluster -- its age,
metallicity, total stellar mass, and reddening by dust among many
others -- is contained in the detailed spectrum and the broadband SED
of the cluster. Our code performs inference of these star cluster
parameters by comparing model spectra and photometry to observations
for a broad range of model parameters.  This is accomplished in the
framework of a likelihood function for the data given the model
parameters.  

The high dimensionality of the parameter space, the desire to robustly
infer degeneracies between the parameters, the presence of informative
prior information about the parameters, and the need marginalize over
``nuisance'' parameters all lead us to a Bayesian methodology based on
Markov Chain Monte Carlo (MCMC) sampling of the likelihood
function. The combination of photometric and spectroscopic information
requires a flexible noise model to account for spectroscopic
calibration uncertainties.

The heart of our star cluster modeling code consists of the
\texttt{FSPS} stellar population synthesis code \citep{fsps}, combined
with a flexible spectroscopic calibration model.  For the MCMC
sampling we are using the affine-invariant ensemble sampler
\texttt{emcee} \citep{emcee} which enables the likelihood function
evaluations to parallelized across a large number CPUs. The main
calculations of \texttt{FSPS} are done in Fortran, but the code is
wrapped in Python.  \texttt{emcee} is written in Python and uses
\texttt{mpi4py} to distribute likelihood calcualtions across CPUs.

\subsection{Star cluster model}
For each set of model parameters a star cluster model must be
generated and compared to the data. The star cluster model is
comprised of a physical model for the spectrum and broadband SED of
and a model for the instrumental calibration.  The physical model is
generated using the FSPS stellar population synthesis code, which
combines tabulated model stellar spectra according to weights
determined from tabulated stellar evolutionary tracks and an initial
mass function.  The FSPS code additionally calculates the reddening
due to dust and convolves the spectrum with a broadening function
representing the instrumental spectral resolution.

The model generation takes approximately 1s on 

\subsubsection{Calibration modeling}
The absolute flux calibration of spectroscopic data is generally
inferior to that of photometric data.  While spectroscopic features
such as absorption line depths provide substantial information about
the stellar population parameters, large scale features in the
observed spectrum and its overall normalization are often affected by
substantial (multiplicative) calibration uncertainties. In order to
simultaneously model the observed spectra and the photometry we have
included a very flexible spectroscopic calibration model based on a
combination of a low-order polynomial function of wavelength and a
Gaussian Process.

The low order polynomial corresponds to gross features in the
calibration as a function of wavelength, while the Gaussian Process
corresponds to smaller scale deviations away from the polynomial,
which can be thought of as covariant uncertainties on the fluxes of
datapoints close in wavelength.

Determining the likelihood in the presence of the covariant
uncertainties requires inverting $N \times N$ matrices, where $N$ is
the number of wavelength points.  This is accomplished using Cholesky
decompositions, as implemented in \texttt{Scipy}.  For our data,
N$\sim 2500$.  On Stampede the matrix inversion takes approximately 1s
on a single processor, using the linked Intel MKL library.

\subsection{MCMC}
For our MCMC sampling we are using the code \texttt{emcee}.  This code
implements an affine-invariant ensemble sampler \citep{goodman_weare}
which is based on an ensemble of ``walkers'' in parameter space that,
at each iteration of the sampler, are used to generate new proposal
positions in parameter space.  This eliminates the need for
hand-tuning of the proposal length scales.  Additionally, the
likelihood calculations for the different walkers at a given iteration
can be simply parallelized.

Parallelization is accomplished as follows. The parameter position of
each walker is sent to a separate processor for the likelihood
evaluation. The likelihoods are then collected and used to generate
new proposed parameter positions for each walker, and a new iteration
is begun.  The density of walker positions constitutes an
approximation to the posterior probability distribution for the
parameters.

Since the run time of our code is dominated by the likelihood
calculations, this parallelization scheme results in very good scaling
with the number of processors, as long as the number of walkers is
larger than the number of processors.  The only data being
communicated between processors are vectors of parameter positions and
the corresponding likelihood (scalars).


\section{Computational Procedure}

\subsection{Observational Data}
Preprocessing and uploading to Stampede

\subsection{Job submission}
We will submit a separate job for each cluster spectrum. normal queue

\subsection{Job details}
Each CPU loads a copy of the likelihood function into memory,
including the FSPS star cluster modelling code and the spectroscopic
calibration model.

For each spectrum, an initial round of optimization of the likelihood
function (actually the posterior probability function) is used to find
the approximate maximum likelihood. This optimization uses Powell's
method as implemented in \texttt{Scipy}, though we are investigating
more efficient algorithms. The optimization is started from as many
positions as there are CPUs available in the job, with each CPU
performing one realization of the optimization.  Thus while increasing
the number of processors does not speed up this phase of the code, it
does substantially reduce the chance of becoming tapped in a local
minimum.

After the separate optimizations have each converged or reached a
maximum number of iterations, the parameters corresponding to the
global minimum are used as a central location for the MCMC sampling.
An ensemble of ``walkers'' is then generated centered on this location
and evolved forward.  The resulting 

\subsection{Post-Processing}
The stored walker positions are transferred to local servers to run
our custom diagnostic and visualization software.  In cases where the
walkers in the sampler have not converged to a stationary distribution
in parameter space, we can restart the sampler from the last ensemble
of positions, thus taking advantage of previous computations.  We are
investigating techniques for robust online assesment of sampler
convergence.


\section{Requested Resources}
Our tests indicate that fro each spectrum approximately X iterations
of Y walkers are required for convergence of a sampler




% \end{deluxetable}

\clearpage

\section{Team}

There are three primary team members working on this project:

\textbf{Daniel Weisz:} \\

\textbf{Benjamin Johnson:} \\

\textbf{Charlie Conroy:}



\bibliographystyle{apj}
\bibliography{references}



\end{document}
